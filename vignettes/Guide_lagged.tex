% Created 2018-08-08 Wed 17:11
% Intended LaTeX compiler: pdflatex
\documentclass[a4paper,twoside,11pt,nojss,article]{jss}
	       \usepackage[T1]{fontenc}
	       \usepackage[left=2cm,right=2cm,bottom=15mm]{geometry}
	       \usepackage{graphicx,color,alltt}
	       \usepackage[authoryear,round,longnamesfirst]{natbib}
	       \usepackage{hyperref}
                              % \usepackage{Sweave}
\author{Georgi N. Boshnakov}
\Plainauthor{Georgi N. Boshnakov}
\Address{
Georgi N. Boshnakov\\
School of Mathematics\\
The University of Manchester\\
Oxford Road, Manchester M13 9PL, UK\\
URL: \url{http://www.maths.manchester.ac.uk/~gb/}
}
\Abstract{
Package \pkg{lagged} provides classes and methods for objects,
whose indexing naturally starts from zero.
\par
This vignette is part of package \pkg{lagged}, version~0.2-0.
}
\Keywords{lag, autocorrelation, indexing}
\Plainkeywords{lag, autocorrelation, indexing}
\author{Georgi N. Boshnakov}
\date{\today}
\title{Inserting figures and evaluated examples}
\hypersetup{
 pdfauthor={Georgi N. Boshnakov},
 pdftitle={Inserting figures and evaluated examples},
 pdfkeywords={examples, graphics, figures, Rd, R},
 pdfsubject={},
 pdfcreator={Emacs 25.3.1 (Org mode 9.1.6)}, 
 pdflang={English}}
\begin{document}

\maketitle
%\SweaveOpts{engine=R,eps=FALSE}

%\VignetteIndexEntry{Inserting figures and evaluated examples}
%\VignetteDepends{lagged}
%\VignetteKeywords{examples, graphics, figures, Rd, R}
%\VignettePackage{lagged}


\section{Univariate lagged objects}
\label{sec:org2fecd57}

Create a univariate lagged object\footnote{The datasets \texttt{ldeaths}, \texttt{fdeaths} and \texttt{mdeaths} are in base R. The examples
involving them are adapted from the help page of \texttt{acf()}.}:
\begin{Schunk}
\begin{Sinput}
> a1 <-  drop(acf(ldeaths)$acf)
> la1 <- Lagged(a1)
> la1
\end{Sinput}
\begin{Soutput}
An object of class "Lagged1d"
Slot *data*: 
       Lag_0        Lag_1        Lag_2        Lag_3        Lag_4        Lag_5 
 1.000000000  0.755051141  0.396956836  0.019395714 -0.355897989 -0.608566374 
       Lag_6        Lag_7        Lag_8        Lag_9       Lag_10       Lag_11 
-0.681383469 -0.607909875 -0.378212377 -0.012975866  0.383252644  0.650206704 
      Lag_12       Lag_13       Lag_14       Lag_15       Lag_16       Lag_17 
 0.723167071  0.638001465  0.371577811  0.009467461 -0.293699737 -0.496742216 
      Lag_18 
-0.585558984 
\end{Soutput}
\end{Schunk}


\texttt{maxLag()} returns the maximal lag in the object. \texttt{Length()} returns the number of lags in the
object, i.e. \texttt{length(la1) == maxLag(la1) + 1}. This relation is a definition and holds also
for multivariate lagged objects. In particular, the length is not necessarily the
length of the data slot.
\begin{Schunk}
\begin{Sinput}
> maxLag(la1)
\end{Sinput}
\begin{Soutput}
[1] 18
\end{Soutput}
\begin{Sinput}
> length(la1)
\end{Sinput}
\begin{Soutput}
[1] 19
\end{Soutput}
\end{Schunk}


\section{Indexing}
\label{sec:org634f5f6}

Indexing drops the "laggedness" to allow easy access to the underlying data\footnote{For some indices, such as \texttt{0:4}, it is possible to keep a Lagged class but it would be
confusing if the indexing operation was returning Lagged or non-Lagged objects depending on
the values of the index.}:
\begin{Schunk}
\begin{Sinput}
> la1[0]
\end{Sinput}
\begin{Soutput}
[1] 1
\end{Soutput}
\begin{Sinput}
> la1[0:4]
\end{Sinput}
\begin{Soutput}
[1]  1.00000000  0.75505114  0.39695684  0.01939571 -0.35589799
\end{Soutput}
\begin{Sinput}
> la1[c(1,3,5)]
\end{Sinput}
\begin{Soutput}
[1]  0.75505114  0.01939571 -0.60856637
\end{Soutput}
\begin{Sinput}
> la1[]
\end{Sinput}
\begin{Soutput}
 [1]  1.000000000  0.755051141  0.396956836  0.019395714 -0.355897989
 [6] -0.608566374 -0.681383469 -0.607909875 -0.378212377 -0.012975866
[11]  0.383252644  0.650206704  0.723167071  0.638001465  0.371577811
[16]  0.009467461 -0.293699737 -0.496742216 -0.585558984
\end{Soutput}
\end{Schunk}


\begin{Schunk}
\begin{Sinput}
> la1a <- la1
> la1a[] <- round(la1, 2)
> la1a
\end{Sinput}
\begin{Soutput}
An object of class "Lagged1d"
Slot *data*: 
 Lag_0  Lag_1  Lag_2  Lag_3  Lag_4  Lag_5  Lag_6  Lag_7  Lag_8  Lag_9 Lag_10 
  1.00   0.76   0.40   0.02  -0.36  -0.61  -0.68  -0.61  -0.38  -0.01   0.38 
Lag_11 Lag_12 Lag_13 Lag_14 Lag_15 Lag_16 Lag_17 Lag_18 
  0.65   0.72   0.64   0.37   0.01  -0.29  -0.50  -0.59 
\end{Soutput}
\end{Schunk}

\begin{Schunk}
\begin{Sinput}
> la1b <- round(la1, 2)
> all(la1a == la1b)
\end{Sinput}
\begin{Soutput}
[1] TRUE
\end{Soutput}
\end{Schunk}




\section{Unary arithmetic and mathematical functions}
\label{sec:orgd6be8cc}

Unary arithmetic operations and mathematical functions replace the data part of the object
and keep its class.
\begin{Schunk}
\begin{Sinput}
> -la1a
\end{Sinput}
\begin{Soutput}
An object of class "Lagged1d"
Slot *data*: 
 Lag_0  Lag_1  Lag_2  Lag_3  Lag_4  Lag_5  Lag_6  Lag_7  Lag_8  Lag_9 Lag_10 
 -1.00  -0.76  -0.40  -0.02   0.36   0.61   0.68   0.61   0.38   0.01  -0.38 
Lag_11 Lag_12 Lag_13 Lag_14 Lag_15 Lag_16 Lag_17 Lag_18 
 -0.65  -0.72  -0.64  -0.37  -0.01   0.29   0.50   0.59 
\end{Soutput}
\begin{Sinput}
> +la1a
\end{Sinput}
\begin{Soutput}
An object of class "Lagged1d"
Slot *data*: 
 Lag_0  Lag_1  Lag_2  Lag_3  Lag_4  Lag_5  Lag_6  Lag_7  Lag_8  Lag_9 Lag_10 
  1.00   0.76   0.40   0.02  -0.36  -0.61  -0.68  -0.61  -0.38  -0.01   0.38 
Lag_11 Lag_12 Lag_13 Lag_14 Lag_15 Lag_16 Lag_17 Lag_18 
  0.65   0.72   0.64   0.37   0.01  -0.29  -0.50  -0.59 
\end{Soutput}
\begin{Sinput}
> ## Math group
> abs(la1a)
\end{Sinput}
\begin{Soutput}
An object of class "Lagged1d"
Slot *data*: 
 Lag_0  Lag_1  Lag_2  Lag_3  Lag_4  Lag_5  Lag_6  Lag_7  Lag_8  Lag_9 Lag_10 
  1.00   0.76   0.40   0.02   0.36   0.61   0.68   0.61   0.38   0.01   0.38 
Lag_11 Lag_12 Lag_13 Lag_14 Lag_15 Lag_16 Lag_17 Lag_18 
  0.65   0.72   0.64   0.37   0.01   0.29   0.50   0.59 
\end{Soutput}
\begin{Sinput}
> sinpi(la1a)
\end{Sinput}
\begin{Soutput}
An object of class "Lagged1d"
Slot *data*: 
      Lag_0       Lag_1       Lag_2       Lag_3       Lag_4       Lag_5 
 0.00000000  0.68454711  0.95105652  0.06279052 -0.90482705 -0.94088077 
      Lag_6       Lag_7       Lag_8       Lag_9      Lag_10      Lag_11 
-0.84432793 -0.94088077 -0.92977649 -0.03141076  0.92977649  0.89100652 
     Lag_12      Lag_13      Lag_14      Lag_15      Lag_16      Lag_17 
 0.77051324  0.90482705  0.91775463  0.03141076 -0.79015501 -1.00000000 
     Lag_18 
-0.96029369 
\end{Soutput}
\begin{Sinput}
> sqrt(abs(la1a))
\end{Sinput}
\begin{Soutput}
An object of class "Lagged1d"
Slot *data*: 
    Lag_0     Lag_1     Lag_2     Lag_3     Lag_4     Lag_5     Lag_6     Lag_7 
1.0000000 0.8717798 0.6324555 0.1414214 0.6000000 0.7810250 0.8246211 0.7810250 
    Lag_8     Lag_9    Lag_10    Lag_11    Lag_12    Lag_13    Lag_14    Lag_15 
0.6164414 0.1000000 0.6164414 0.8062258 0.8485281 0.8000000 0.6082763 0.1000000 
   Lag_16    Lag_17    Lag_18 
0.5385165 0.7071068 0.7681146 
\end{Soutput}
\begin{Sinput}
> ## Math2 group
> round(la1a)
\end{Sinput}
\begin{Soutput}
An object of class "Lagged1d"
Slot *data*: 
 Lag_0  Lag_1  Lag_2  Lag_3  Lag_4  Lag_5  Lag_6  Lag_7  Lag_8  Lag_9 Lag_10 
     1      1      0      0      0     -1     -1     -1      0      0      0 
Lag_11 Lag_12 Lag_13 Lag_14 Lag_15 Lag_16 Lag_17 Lag_18 
     1      1      1      0      0      0      0     -1 
\end{Soutput}
\begin{Sinput}
> round(la1a, 2)
\end{Sinput}
\begin{Soutput}
An object of class "Lagged1d"
Slot *data*: 
 Lag_0  Lag_1  Lag_2  Lag_3  Lag_4  Lag_5  Lag_6  Lag_7  Lag_8  Lag_9 Lag_10 
  1.00   0.76   0.40   0.02  -0.36  -0.61  -0.68  -0.61  -0.38  -0.01   0.38 
Lag_11 Lag_12 Lag_13 Lag_14 Lag_15 Lag_16 Lag_17 Lag_18 
  0.65   0.72   0.64   0.37   0.01  -0.29  -0.50  -0.59 
\end{Soutput}
\begin{Sinput}
> signif(la1a)
\end{Sinput}
\begin{Soutput}
An object of class "Lagged1d"
Slot *data*: 
 Lag_0  Lag_1  Lag_2  Lag_3  Lag_4  Lag_5  Lag_6  Lag_7  Lag_8  Lag_9 Lag_10 
  1.00   0.76   0.40   0.02  -0.36  -0.61  -0.68  -0.61  -0.38  -0.01   0.38 
Lag_11 Lag_12 Lag_13 Lag_14 Lag_15 Lag_16 Lag_17 Lag_18 
  0.65   0.72   0.64   0.37   0.01  -0.29  -0.50  -0.59 
\end{Soutput}
\begin{Sinput}
> signif(la1a, 4)
\end{Sinput}
\begin{Soutput}
An object of class "Lagged1d"
Slot *data*: 
 Lag_0  Lag_1  Lag_2  Lag_3  Lag_4  Lag_5  Lag_6  Lag_7  Lag_8  Lag_9 Lag_10 
  1.00   0.76   0.40   0.02  -0.36  -0.61  -0.68  -0.61  -0.38  -0.01   0.38 
Lag_11 Lag_12 Lag_13 Lag_14 Lag_15 Lag_16 Lag_17 Lag_18 
  0.65   0.72   0.64   0.37   0.01  -0.29  -0.50  -0.59 
\end{Soutput}
\end{Schunk}

The functions from the summary group work on the data part, as if they were called on it.
\begin{Schunk}
\begin{Sinput}
> c(Max = max(la1a), Min = min(la1a), Range = range(la1a))
\end{Sinput}
\begin{Soutput}
   Max    Min Range1 Range2 
  1.00  -0.68  -0.68   1.00 
\end{Soutput}
\begin{Sinput}
> c(Prod = prod(la1a), Sum = sum(la1a))
\end{Sinput}
\begin{Soutput}
         Prod           Sum 
-7.582098e-11  9.200000e-01 
\end{Soutput}
\begin{Sinput}
> c(Any = any(la1a < 0), All = all(la1a >= 0))
\end{Sinput}
\begin{Soutput}
  Any   All 
 TRUE FALSE 
\end{Soutput}
\end{Schunk}

Binary arithmetic operators are defined between two lagged objects and between a lagged
object and a vector. They return a lagged object from one of the "basic" lagged classes, but
not necessarilly exactly from the class of the argument(s). The class of the returned value
is from a suitable lagged superclass of the argument(s). This concerns operations on objects
from classes inheriting from the classes considered here, so is not visible in the examples
below, since they use objects from the basic lagged classes.
\begin{Schunk}
\begin{Sinput}
> 2*la1a
\end{Sinput}
\begin{Soutput}
An object of class "Lagged1d"
Slot *data*: 
 Lag_0  Lag_1  Lag_2  Lag_3  Lag_4  Lag_5  Lag_6  Lag_7  Lag_8  Lag_9 Lag_10 
  2.00   1.52   0.80   0.04  -0.72  -1.22  -1.36  -1.22  -0.76  -0.02   0.76 
Lag_11 Lag_12 Lag_13 Lag_14 Lag_15 Lag_16 Lag_17 Lag_18 
  1.30   1.44   1.28   0.74   0.02  -0.58  -1.00  -1.18 
\end{Soutput}
\begin{Sinput}
> la1a^2
\end{Sinput}
\begin{Soutput}
An object of class "Lagged1d"
Slot *data*: 
 Lag_0  Lag_1  Lag_2  Lag_3  Lag_4  Lag_5  Lag_6  Lag_7  Lag_8  Lag_9 Lag_10 
1.0000 0.5776 0.1600 0.0004 0.1296 0.3721 0.4624 0.3721 0.1444 0.0001 0.1444 
Lag_11 Lag_12 Lag_13 Lag_14 Lag_15 Lag_16 Lag_17 Lag_18 
0.4225 0.5184 0.4096 0.1369 0.0001 0.0841 0.2500 0.3481 
\end{Soutput}
\begin{Sinput}
> la1a + la1a^2
\end{Sinput}
\begin{Soutput}
An object of class "Lagged1d"
Slot *data*: 
  Lag_0   Lag_1   Lag_2   Lag_3   Lag_4   Lag_5   Lag_6   Lag_7   Lag_8   Lag_9 
 2.0000  1.3376  0.5600  0.0204 -0.2304 -0.2379 -0.2176 -0.2379 -0.2356 -0.0099 
 Lag_10  Lag_11  Lag_12  Lag_13  Lag_14  Lag_15  Lag_16  Lag_17  Lag_18 
 0.5244  1.0725  1.2384  1.0496  0.5069  0.0101 -0.2059 -0.2500 -0.2419 
\end{Soutput}
\begin{Sinput}
> la1a - la1a^2
\end{Sinput}
\begin{Soutput}
An object of class "Lagged1d"
Slot *data*: 
  Lag_0   Lag_1   Lag_2   Lag_3   Lag_4   Lag_5   Lag_6   Lag_7   Lag_8   Lag_9 
 0.0000  0.1824  0.2400  0.0196 -0.4896 -0.9821 -1.1424 -0.9821 -0.5244 -0.0101 
 Lag_10  Lag_11  Lag_12  Lag_13  Lag_14  Lag_15  Lag_16  Lag_17  Lag_18 
 0.2356  0.2275  0.2016  0.2304  0.2331  0.0099 -0.3741 -0.7500 -0.9381 
\end{Soutput}
\begin{Sinput}
> la1a * la1a^2
\end{Sinput}
\begin{Soutput}
An object of class "Lagged1d"
Slot *data*: 
    Lag_0     Lag_1     Lag_2     Lag_3     Lag_4     Lag_5     Lag_6     Lag_7 
 1.000000  0.438976  0.064000  0.000008 -0.046656 -0.226981 -0.314432 -0.226981 
    Lag_8     Lag_9    Lag_10    Lag_11    Lag_12    Lag_13    Lag_14    Lag_15 
-0.054872 -0.000001  0.054872  0.274625  0.373248  0.262144  0.050653  0.000001 
   Lag_16    Lag_17    Lag_18 
-0.024389 -0.125000 -0.205379 
\end{Soutput}
\begin{Sinput}
> la1a / la1a^2
\end{Sinput}
\begin{Soutput}
An object of class "Lagged1d"
Slot *data*: 
      Lag_0       Lag_1       Lag_2       Lag_3       Lag_4       Lag_5 
   1.000000    1.315789    2.500000   50.000000   -2.777778   -1.639344 
      Lag_6       Lag_7       Lag_8       Lag_9      Lag_10      Lag_11 
  -1.470588   -1.639344   -2.631579 -100.000000    2.631579    1.538462 
     Lag_12      Lag_13      Lag_14      Lag_15      Lag_16      Lag_17 
   1.388889    1.562500    2.702703  100.000000   -3.448276   -2.000000 
     Lag_18 
  -1.694915 
\end{Soutput}
\begin{Sinput}
> la1a + 1:length(la1a)
\end{Sinput}
\begin{Soutput}
An object of class "Lagged1d"
Slot *data*: 
 Lag_0  Lag_1  Lag_2  Lag_3  Lag_4  Lag_5  Lag_6  Lag_7  Lag_8  Lag_9 Lag_10 
  2.00   2.76   3.40   4.02   4.64   5.39   6.32   7.39   8.62   9.99  11.38 
Lag_11 Lag_12 Lag_13 Lag_14 Lag_15 Lag_16 Lag_17 Lag_18 
 12.65  13.72  14.64  15.37  16.01  16.71  17.50  18.41 
\end{Soutput}
\end{Schunk}

There is a case to argue for keeping the class in some situations, e.g. when the other
argument is a scalar but eventually I decided to keep the simple rule of not trying to
preserve the class. 

Note however that unary operators and mathematical functions do preserve the class.

\section{Multivariate lagged objects}
\label{sec:org4e80c18}

Compute the autocorrelations of a multivariate time series and convert it to a lagged object.
\begin{Schunk}
\begin{Sinput}
> acv2 <- acf(ts.union(mdeaths, fdeaths))
> la2 <- Lagged(acv2)
\end{Sinput}
\end{Schunk}

Get the value for lag 1.
\begin{Schunk}
\begin{Sinput}
> la2[1]
\end{Sinput}
\begin{Soutput}
, , 1

          [,1]      [,2]
[1,] 0.7570591 0.7356685
[2,] 0.7443093 0.7295201
\end{Soutput}
\begin{Sinput}
> acv2$acf[2, ,] # same
\end{Sinput}
\begin{Soutput}
          [,1]      [,2]
[1,] 0.7570591 0.7356685
[2,] 0.7443093 0.7295201
\end{Soutput}
\end{Schunk}

Indexing in \texttt{acf()} is somewhat misterious. For some insight, here is a comparison with a DIY
calculation of the autocorrelations.
\begin{Schunk}
\begin{Sinput}
> n <- length(mdeaths)
> tmpcov <- sum((mdeaths - mean(mdeaths)) * (fdeaths - mean(fdeaths)) ) / n
> msd <- sqrt(sum((mdeaths - mean(mdeaths))^2)/n)
> fsd <- sqrt(sum((fdeaths - mean(fdeaths))^2)/n)
> tmpcov1 <- sum((mdeaths - mean(mdeaths))[2:n] * (fdeaths - mean(fdeaths))[1:(n-1)] ) / n
> tmpcov1 / (msd * fsd)
\end{Sinput}
\begin{Soutput}
[1] 0.7356685
\end{Soutput}
\begin{Sinput}
> la2[[1]][1,2] == tmpcov1 / (msd * fsd) # FALSE, but:
\end{Sinput}
\begin{Soutput}
[1] FALSE
\end{Soutput}
\begin{Sinput}
> la2[[1]][1,2] - tmpcov1 / (msd * fsd)  # only numerically different
\end{Sinput}
\begin{Soutput}
[1] 2.220446e-16
\end{Soutput}
\end{Schunk}

Some examples for the correspondence between the indices in lagged objects and those from
\texttt{acf()}.
\begin{Schunk}
\begin{Sinput}
> la2[[1]][1,2] == acv2$acf[2, 1, 2] # TRUE
\end{Sinput}
\begin{Soutput}
[1] TRUE
\end{Soutput}
\begin{Sinput}
> la2[0]
\end{Sinput}
\begin{Soutput}
, , 1

          [,1]      [,2]
[1,] 1.0000000 0.9762413
[2,] 0.9762413 1.0000000
\end{Soutput}
\begin{Sinput}
> acv2[0]
\end{Sinput}
\begin{Soutput}
Autocorrelations of series �ts.union(mdeaths, fdeaths)�, by lag

, , mdeaths

 mdeaths   fdeaths  
 1.000 (0) 0.976 (0)

, , fdeaths

 mdeaths   fdeaths  
 0.976 (0) 1.000 (0)
\end{Soutput}
\begin{Sinput}
> la2[1]
\end{Sinput}
\begin{Soutput}
, , 1

          [,1]      [,2]
[1,] 0.7570591 0.7356685
[2,] 0.7443093 0.7295201
\end{Soutput}
\begin{Sinput}
> acv2[1]
\end{Sinput}
\begin{Soutput}
Autocorrelations of series �ts.union(mdeaths, fdeaths)�, by lag

, , mdeaths

 mdeaths    fdeaths   
 0.717 ( 1) 0.708 (-1)

, , fdeaths

 mdeaths    fdeaths   
 0.721 ( 1) 0.716 ( 1)
\end{Soutput}
\end{Schunk}
\end{document}
