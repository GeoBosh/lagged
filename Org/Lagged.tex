% Created 2019-03-31 Sun 21:42
% Intended LaTeX compiler: pdflatex
\documentclass[11pt,a4paper]{article}
	      \usepackage[left=1.8cm,right=1.8cm,top=2cm,bottom=2cm]{geometry}
\usepackage[utf8]{inputenc}
\usepackage[T1]{fontenc}
\usepackage{graphicx}
\usepackage{grffile}
\usepackage{longtable}
\usepackage{wrapfig}
\usepackage{rotating}
\usepackage[normalem]{ulem}
\usepackage{amsmath}
\usepackage{textcomp}
\usepackage{amssymb}
\usepackage{capt-of}
\usepackage{hyperref}
\author{Georgi N. Boshnakov}
\date{\today}
\title{Lagged objects in package 'lagged'}
\hypersetup{
 pdfauthor={Georgi N. Boshnakov},
 pdftitle={Lagged objects in package 'lagged'},
 pdfkeywords={},
 pdfsubject={},
 pdfcreator={Emacs 25.3.1 (Org mode 9.1.6)}, 
 pdflang={English}}
\begin{document}

\maketitle
\tableofcontents

\vspace*{1cm}
This message is printed at the start of the tangled .R file to alert against editing that
file:
\begin{verbatim}
## Do not edit this file manually.
## It has been automatically generated from *.org sources.
\end{verbatim}



\section{Class "Lagged"}
\label{sec:orgfa7184e}


"Lagged" is the base lagged class. It is virtual and defines a slot \texttt{data} from class
"ANY".
\begin{verbatim}
setClass("Lagged", slots = c(data = "ANY"), contains = "VIRTUAL")
                               # setClass("Lagged", slots = c(data = "vector") )
                               # setClass("Lagged", slots = c(data = "structure") )
\end{verbatim}

Actual classes inherit from "Lagged" and restrict the data slot.  One special subclass
of "Lagged" is "FlexibleLagged", which can represent objects from any subclass of
"Lagged". This is achieved by setting  the data slot to be "Lagged" along with methods for
indexing, \texttt{maxLag}, and similar to ensure that the Lagged functionality is provided.
\begin{verbatim}
setClass("FlexibleLagged", contains = "Lagged", slots = c(data = "Lagged") )
\end{verbatim}
The initialisation method for "FlexibleLagged" objects does the obvious thing if argument
\texttt{data} is "Lagged". Otherwise it converts argument \texttt{data} to a suitable "Lagged" object
before assigning it to the data slot. If the function is not able to infer a suitable
"Lagged" class it still passes \texttt{data} on to the next method which usually leads to an error.

The following convenience function is used to infer a suitable "Lagged" class for argument
\texttt{data}:
\begin{verbatim}
.whichNativeLagged <- function(x){
    if(is(x, "Lagged"))
        "FlexibleLagged"
    else if(is.vector(x))
        "Lagged1d"
    else if(is.matrix(x))
        "Lagged2d"
    else if(is.array(x)  && length(dim(x)) == 3)
        "Lagged3d"
    else
        NA

}
\end{verbatim}


This is the initialisation method for \texttt{"FlexibleLagged"}. Note that it gets rid of recursive
\texttt{"FlexibleLagged"} data slots, i.e. the data slot of the returned object is "Lagged" but not
"FlexibleLagged".  This probably should be taken care of by a validation method.
\begin{verbatim}
setMethod("initialize", "FlexibleLagged",
          function(.Object, data, ...){
              if(missing(data))
                  return(callNextMethod(.Object, ...))

              while(is(data, "FlexibleLagged"))
                  data <- data@data

              if(!is(data, "Lagged")){
                  clname <- .whichNativeLagged(data)
                  if(!is.na(clname))
                      data <- new(clname, data = data)
                  ##else don't know what to do with data, pass it on
                  ##     and let others complain if not appropriate.
              }
              .Object <- callNextMethod(.Object, data = data, ...)

              .Object
          }
          )
\end{verbatim}
In general, "FlexibleLagged" can be used as superclass of classes which wish to represent any
possible subclasses of "Lagged". For slots, it is sufficient (and more efficient) to use
"Lagged".

Since class "FlexibleLagged" is special, it has its own implementations of some core methods
defined for "Lagged".

\textbf{TODO:} Decide what support to offer for the native S3 class "acf". Turn it into S4 using
\texttt{setOldClass}? Or just adapt the various methods and constructors to convert it properly to
Lagged? For now putting some code in \texttt{Lagged()} to accept "acf" objects.

\subsection{Core methods for lagged objects}
\label{sec:org6d089da}

The methods in this section are ok for objects inheriting from "Lagged". Where
necessary, specialised methods are defined for "FlexibleLagged".


\subsubsection{Subscripting with "["}
\label{sec:org18c9cac}

Subscripting with \texttt{i} missing, returns the raw data.
\begin{verbatim}
setMethod("[", c(x = "Lagged", i = "missing"), function(x) x@data )
setMethod("[", c(x = "FlexibleLagged", i = "missing"), function(x) x@data[] )
\end{verbatim}

When \texttt{i} is present, indexing depends on the type of the data slot and so is defined by
subclasses. For indices larger than \texttt{maxLag(x)} the values are filled with NA's.


\textbf{TODO:} consider making the 1d method the default one?

For "[", the default for \texttt{drop} is \texttt{FALSE}.
\textbf{TODO:} check that the existing methods follow this convention!

Indexing "FlexibleLagged" simply transfers the operation to the data slot (it is "Lagged"):
\begin{verbatim}
setMethod("[", c(x = "FlexibleLagged"), function(x, i, ...) x@data[i, ...] )
\end{verbatim}


\subsubsection{Subscript-replacement with "[<-"}
\label{sec:orgc5dad1d}

Similarly to "[", subscript-replacement "[<-" replaces the contents of the data.  The method
for "Lagged" does not check the validity of argument \texttt{value} but the assignment will
raise an error if it is not appropriate. Subclasses that wish to provide finer control over
this can define suitable methods (e.g. to coerce \texttt{value} appropriately).
\begin{verbatim}
setReplaceMethod("[", c(x = "Lagged", i = "missing"),
          function(x, i, value){
              x@data <- value
              x
          })
\end{verbatim}

Assignment to "FlexibleLagged", when \texttt{i} is missing, attempts to coerce \texttt{value} to a suitable
lagged class before assigning it (using \texttt{.whichNativeLagged()}, as the initialisation
function does, but raising an error if unsuccessful). Further methods can be defined using
\texttt{"value = xxx"} in the signature to accommodate additional types or overwrite the default
method here.
\begin{verbatim}
setReplaceMethod("[", c(x = "FlexibleLagged", i = "missing"),
                 function(x, i, value){
                     if(is(value, "FlexibleLagged"))
                         x@data <- value@data
                     else if(is(value, "Lagged"))
                         x@data <- value
                     else{
                         clname <- .whichNativeLagged(value)
                         if(is.na(clname))
                             stop("Don't know what Lagged class to use for this value")
                         else
                             x@data <- new(clname, data = value) # as(value, clname)
                     }
                     x
                 })
\end{verbatim}
When \texttt{i} is present, no attempt is made to coerce it:
\begin{verbatim}
setReplaceMethod("[", c(x = "FlexibleLagged", i = "numeric"),
                 function(x, i, value){
                     x@data[i] <- value # not i+1, since x@data is a "Lagged" object here.
                     x
          })
\end{verbatim}

\begin{verbatim}
## Ne, tezi zasega ne gi pravya, pravya vischko bez "value = xxx" - tova pozvolyava da se
## definirat metodi ako tryabva.
##
## setReplaceMethod("[", c(x = "FlexibleLagged", i = "missing", value = "vector"),
##           function(x, i, value){
##               x@data <- as(value, "Lagged1d")
##               x
##           })
##
## setReplaceMethod("[", c(x = "FlexibleLagged", i = "missing", value = "matrix"),
##           function(x, i, value){
##               x@data <- as(value, "Lagged2d")
##               x
##           })
\end{verbatim}


\subsubsection{Methods for "[[" and "[[<-"}
\label{sec:orgb507196}

Indexing with "[[" returns the value for the specified lag. This is the recommended way to
extract the value at a single index.

This defines a default method. For efficiency specific classes can define versions that avoid
calling the generic "[[". If multi-seasons are supported the check for length equal to one
should be adapted accordingly.
\begin{verbatim}
setMethod("[[", c(x = "Lagged", i = "numeric"),
          function(x, i){
              if(length(i) == 1)
                  x[i, drop = TRUE]
              else
                  stop("length of argument `i' must be equal to one")
          }
          )
\end{verbatim}
Note the use of \texttt{drop = TRUE}.

\textbf{TODO:} The use of \texttt{drop = TRUE} maybe needs some further thought. Maybe something that drops
only the index corresponding to the lag is preferable and such behaviour should be documented!

The replace method works similarly:
\begin{verbatim}
setReplaceMethod("[[", c(x = "Lagged", i = "numeric"),
                 function(x, i, value){
                     if(length(i) == 1)
                         x[i] <- value
                     else
                         stop("length of argument `i' must be equal to one")
                     x
                 })
\end{verbatim}



\subsubsection{Arithmetic and other operations (Ops group)}
\label{sec:org8bd6503}

Operations in the \texttt{Ops} group involving lagged objects are defined "naturally" on their
data. However, they are more restrictive than base R's conventions for atomic objects and do
not follow the recycling rules.

The binary "Ops" methods return values from one of the core lagged classes, even if the
objects are from classes inheriting from "Lagged". The reason is that, for example, the
difference between autocovariance functions is not necessarilly autocovariance, but it is
still a lagged object. It would be very confusing if the result was not guaranteed to be
"Lagged".  Also, if a policy of preserving the actual class were to be adopted, what would
be the rule for the class of the result from binary operations between lagged objects from
different classes (it seems not possible to have a simple one). 


Of course, methods defined for subclasses of lagged objects may preserve the actual classes
when appropriate but should not introduce confusion on indexing.

In the default methods below, the result of these operations is a strict lagged object,
i.e. an object from the core lagged classes (\textbf{TODO:} explain). The exact type of lagged
object is determined by the data. The net effect is that the value of the Ops operation is
also a lagged object, a core one, with indexing starting from zero but additional structure
is lost.


\textbf{TODO:} Should operations between "Lagged" and base R objects be permitted at all?  For users
of "Lagged" the returned "Lagged" value is natural and expected. But what about users who are
not aware that there are "Lagged" objects among the arguments? What to do when the "ordinary"
argument is of length one - should this be an exception? But then the user may not know that
the length is one, leading to surprises. Also, there is a conceptual difference here between
the additive and multiplicative operations. (All this should be documented in a vignette. It
seems sufficient that the recycling rule is banned. Need to finalise operation with
singletons.)


Operations between two lagged objects give a lagged object. If their \texttt{maxLag()} properties
are different, the shorter data slot is extended with NA's before applying the binary
operation.


\begin{enumerate}
\item "Ops" involving "Lagged"
\label{sec:orgee8dc55}

The unary operators preserve the class of the object:
\begin{verbatim}
setMethod("Ops", c(e1 = "Lagged", e2 = "missing"),
          function(e1){
                    # wrk <- callGeneric(e1@data)
                    # clname <- whichLagged(e1)
                    # new(clname, data = wrk)
              e1@data <- callGeneric(e1@data)
              e1
          })
\end{verbatim}

\begin{verbatim}
## TODO: do not allow mixing Lagged1d with Lagged2d, etc.?
setMethod("Ops", c(e1 = "Lagged", e2 = "Lagged"),
          function(e1, e2){
              wrk <- if(length(e1@data) == length(e2@data) ) # TODO: allow %%==0 as elsewhere?
                         callGeneric(e1@data, e2@data)
                     else{
                         maxlag <- max(maxLag(e1), maxLag(e2))
                         v1 <- e1[0:maxlag]
                         v2 <- e2[0:maxlag]
                         callGeneric(v1, v2)
                     }
              clname <- whichLagged(e1, e2)
              new(clname, data = wrk)
          })
\end{verbatim}
\textbf{TODO:} the current mechanism to decide the lagged class of the return value is not very
satisfactory, see \texttt{whichLagged()} which encapsulates it. Also, forbid mixing 1d with 2d,
etc.?

When only one of the objects is "Lagged", the operations are defined if the following cases:

\begin{enumerate}
\item the length of the other object is equal to the length of the data part of the "Lagged"
object,
\item the other object is of length one,
\item the other object is a singleton with the same dimensions as a single element of the
"Lagged" object.
\end{enumerate}

\textbf{old todo:} document behaviour if \texttt{length(object@data) == 0} (minor issue)?

\textbf{2017-05-20 TODO:} Change \texttt{length(e1[[0]]) == length(e2))} below to
                   \texttt{dim(e1[[0]]) == dim(e2))} but needs more care (note though that the
                   scalar case is covered by \texttt{length(e2) == 1}.

Notice that "vector" in the signatures is the S4 class "vector" (TODO: check!), see
\texttt{showClass("vector")} for its subclasses.
\begin{verbatim}
> is.vector(array(0, dim = c(2,2,2)))    # S3
[1] FALSE

> is(array(0, dim = c(2,2,2)), "vector") # S4
[1] TRUE
\end{verbatim}


\begin{verbatim}
setMethod("Ops", c(e1 = "Lagged", e2 = "vector"),
          function(e1, e2){
              wrk <- if(length(e2) == 1  || length(e1@data) == length(e2)
                             # 2017-05-20 was:
                             #    || length(e2) > 0  && (length(e1@data) %% length(e2)) == 0
                        || length(e2) > 0  && length(e1[[0]]) == length(e2))
                         callGeneric(e1@data, e2)
                     else
                         stop("Incompatible length of operands in a binary operation")

              new(whichLagged(e1), data = wrk)
          })

setMethod("Ops", c(e1 = "vector", e2 = "Lagged"),
          function(e1, e2){
              wrk <- if(length(e1) == 1  || length(e1) == length(e2@data)
                             # 2017-05-20 was:
                             #    || length(e1) > 0  && (length(e2@data) %% length(e1)) == 0
                        || length(e1) > 0  && length(e2[[0]]) == length(e1))
                         callGeneric(e1, e2@data)
                     else
                         stop("Incompatible length of operands in a binary operation")

              new(whichLagged(e2), data = wrk)
          })
\end{verbatim}


\item "Ops" involving "FlexibleLagged"
\label{sec:orgf0aa19e}

Operations involving "FlexibleLagged" objects use those defined for "Lagged" by operating on
the data slot (which is "Lagged").
\begin{verbatim}
setMethod("Ops", c(e1 = "FlexibleLagged", e2 = "Lagged"),
          function(e1, e2){
              callGeneric(e1@data, e2)
          })

setMethod("Ops", c(e1 = "Lagged", e2 = "FlexibleLagged"),
          function(e1, e2){
              callGeneric(e1, e2@data)
          })

setMethod("Ops", c(e1 = "FlexibleLagged", e2 = "FlexibleLagged"),
          function(e1, e2){
              callGeneric(e1@data, e2@data)
          })


setMethod("Ops", c(e1 = "FlexibleLagged", e2 = "vector"),
          function(e1, e2){
              callGeneric(e1@data, e2)
          })

setMethod("Ops", c(e1 = "vector", e2 = "FlexibleLagged"),
          function(e1, e2){
              callGeneric(e1, e2@data)
          })
\end{verbatim}

\textbf{TODO:} methods for "matrix", "array", these probably should be for specific "Lagged"
subclasses, like "Lagged2d".
\end{enumerate}


\subsubsection{"Math" and "Math2" group methods}
\label{sec:org801bdd9}

"Math" and "Math2" methods return the object with its data part transformed by the
corresponding function. 

\textbf{TODO:} Does this work for \texttt{FlexibleLagged}?
\begin{verbatim}
setMethod("Math", c(x = "Lagged"),
          function(x){
              x@data <- callGeneric(x@data)
              x
          })
\end{verbatim}


\begin{verbatim}
setMethod("Math2", c(x = "Lagged"),
          function(x, digits){
              x@data <- callGeneric(x@data, digits)
              x
          })
\end{verbatim}


\subsubsection{"Summary" group methods}
\label{sec:org5162825}

The "Summary" methods operate on the data part of the "Lagged" object.
\begin{verbatim}
setMethod("Summary", c(x = "Lagged"),
          function(x, ..., na.rm = FALSE){
              callGeneric(x@data)
          })
\end{verbatim}


\subsection{S3 methods for as.vector() and related functions for "Lagged"}
\label{sec:orgc390c9a}

\begin{verbatim}
## TODO: check if the S3 methods understand S4 inheritance (I think they do)
as.vector.Lagged <- function(x, mode) as.vector(x@data) # todo: use mode?
as.double.Lagged <- function(x, ...)  as.double(x@data ) # note: this is for as.numeric()
as.matrix.Lagged <- function(x, ...)  as.matrix(x@data)
 as.array.Lagged <- function(x, ...)  as.array(x@data)
\end{verbatim}
Converting from "Lagged" to base atomic or structure objects applies the requested
operation to the data slot. Define first the generic S3 methods:

Somewhat more efficient methods for these:
\begin{verbatim}
as.vector.Lagged1d <- function(x, mode) x@data
as.matrix.Lagged2d <- function(x, ...) x@data
as.array.Lagged3d  <- function(x, ...) x@data
\end{verbatim}


\subsection{setAs() methods for "Lagged"}
\label{sec:org997b2c6}

These methods call the corresponding S3 methods defined above:
\begin{verbatim}
setAs("Lagged", "vector", function(from) as.vector(from) )
setAs("Lagged", "matrix", function(from) as.matrix(from) )
setAs("Lagged", "array",  function(from) as.array(from) )
\end{verbatim}



\subsection{Generic function maxLag()}
\label{sec:orgc699efe}

The default method for \texttt{maxLag()} handles objects inheriting from S3 class "acf". In all
other cases it raises an error. Notice that in "acf" the lag is in the first dimension.
\begin{verbatim}
maxLag <- function(object, ...){
   if(inherits(object, "acf"))
       dim(acf$acf)[1] - 1
   else
       stop("No applicable method to compute maxLag")
}

setGeneric("maxLag")
\end{verbatim}

\begin{verbatim}
setGeneric("maxLag<-", def = function(object, ..., value){ standardGeneric("maxLag<-") } )
\end{verbatim}

\textbf{TODO:} Do we need a separate method for "FlexibleLagged"?
\begin{verbatim}
setReplaceMethod("maxLag", "Lagged",
                 function(object, ..., value){
                     object@data <- object[0:value]
                     object
                 }
                 )
\end{verbatim}

The convention for "Lagged" objects is that the last dimension carries the lag.  So, the
methods for basic objects compute the maximal lag as the last dimention minus one.
\begin{verbatim}
setMethod("maxLag", c(object = "vector"), function(object) length(object) - 1)
setMethod("maxLag", c(object = "matrix"), function(object) ncol(object) - 1 )
setMethod("maxLag", c(object = "array"),
          function(object){
                  d <- dim(object)
                  d[length(d)] - 1
          })
\end{verbatim}
Note again that \texttt{acf()} puts the lag in the first index.

The \texttt{maxLag()} method for "Lagged" objects simply calls \texttt{maxLag()} on the data slot. Classes
inheriting from "Lagged" may define specific methods if the (in)efficiency of this method is
a concern.
\begin{verbatim}
setMethod("maxLag", c(object = "Lagged"), function(object) maxLag(object@data) )
\end{verbatim}


\subsection{Length of "Lagged" objects - S3 method for length()}
\label{sec:org1788c69}

The length of "Lagged" objects is defined to be \texttt{maxLag(x)+1}, not the length of the data in
the "Lagged" object. In most cases of direct use \texttt{maxLag(x)} is more appropriate.

This defines an S3 method for function \texttt{length()} for "Lagged" objects.
\begin{verbatim}
length.Lagged <- function(x) maxLag(x) + 1
\end{verbatim}

\textbf{TODO:} Check if other base R functions need S3 methods for "Lagged" objects.



\section{Default classes and methods for lagged objects}
\label{sec:orgd1a169d}


\begin{verbatim}
                                               # setClass("X", slots = c(data = "structure"))
setClass("Lagged1d", contains = "Lagged", slots = c(data = "vector") )
setClass("Lagged2d", contains = "Lagged", slots = c(data = "matrix") )
setClass("Lagged3d", contains = "Lagged", slots = c(data = "array") )
                     # TODO: check validity for Lagged3d: 3 dimensional.
\end{verbatim}


\subsection{Methods for "["}
\label{sec:org49ffca1}

\begin{verbatim}
setMethod("[", c(x = "Lagged1d", i = "numeric"),
          function(x, i, drop) x@data[i+1] )

## TODO: argument "drop"?
setMethod("[", c(x = "Lagged2d", i = "numeric"),
          function(x, i, drop = FALSE) x@data[ , i+1, drop = drop] )

## TODO: change autocovariances(), etc to this convention!!
setMethod("[", c(x = "Lagged3d", i = "numeric"),
          function(x, i, drop = FALSE) x@data[, , i+1, drop = drop] )
\end{verbatim}

\subsection{whichLagged()}
\label{sec:org99d682b}

For now \texttt{whichLagged()} is not exported. It could be exported to allow core "Lagged" classes
defined in other packages to add functionality. But if it is to be exported, it would need
streamlining. Currently it is a hack.

Making it generic is lazy but avoids writing obscure code but see note above.
The default returns "FlexibleLagged".
\begin{verbatim}
.matLagged <- matrix("FlexibleLagged", 4, 4)
diag(.matLagged) <- c("FlexibleLagged", "Lagged1d", "Lagged2d", "Lagged3d")

rownames(.matLagged) <- c("FlexibleLagged", "Lagged1d", "Lagged2d", "Lagged3d")
colnames(.matLagged) <- c("FlexibleLagged", "Lagged1d", "Lagged2d", "Lagged3d")


whichLagged <- function(x, y){
    .matLagged[whichLagged(x), whichLagged(y)]
}
setGeneric("whichLagged")
\end{verbatim}

\begin{verbatim}
## TODO: define methods for "numeric", "matrix", etc?
setMethod("whichLagged", c(x = "ANY"     , y = "missing"), function(x) "FlexibleLagged")
setMethod("whichLagged", c(x = "Lagged1d", y = "missing"), function(x) "Lagged1d")
setMethod("whichLagged", c(x = "Lagged2d", y = "missing"), function(x) "Lagged2d")
setMethod("whichLagged", c(x = "Lagged3d", y = "missing"), function(x) "Lagged3d")
\end{verbatim}


\subsection{Methods for "[<-"}
\label{sec:org1675942}

Missing index is equivalent to replacing all data: 
\begin{verbatim}
setReplaceMethod("[", c(x = "Lagged", i = "missing"),
          function(x, i, value){
              x[0:maxLag(x)] <- value
              x
          })
\end{verbatim}
The above method just calls "[<-" again, so it applies to any lagged objects.

The methods which work on the data, need to know their layout, so we need several methods.
\begin{verbatim}
setReplaceMethod("[", c(x = "Lagged1d", i = "numeric"),
          function(x, i, value){
              x@data[i+1] <- value
              x
          })

setReplaceMethod("[", c(x = "Lagged2d", i = "numeric"), #Include value = "matrix" in signature?
          function(x, i, value){
              x@data[ , i+1]  <- value
              x
          })

## Include value = "array" in the signature? Will still need to check the dimensions
setReplaceMethod("[", c(x = "Lagged3d", i = "numeric"),
          function(x, i, value){
                      # was: x@data[i+1, , ]  <- value
              x@data[ , , i+1]  <- value
              x
          })
\end{verbatim}




\section{show() methods}
\label{sec:org3e1dbfc}

\begin{verbatim}
## .printVecOrArray <- function(x){
##     if(is.vector(x)){
##         if(is.null(names(x)) || length(names(x)) == 0)
##             names(x) <- paste0("Lag_", 0:(length(x) - 1))
##         print(x)
##     }else if(is.matrix(x)){
##         ## TODO:
##         print(x)
##     }else if(is.array(x)){
##         ## TODO:
##         print(x)
##     }else
##         print(x)
## }
\end{verbatim}


\begin{verbatim}
setMethod("show", "Lagged1d",
          function(object){
              .reportClassName(object, "Lagged1d")
              cat("Slot *data*:", "\n")

              ## 2017-05-24 was:
              ##     x <- object@data
              ##     if(is.null(names(x)) || length(names(x)) == 0)
              ##         names(x) <- paste0("Lag_", 0:(length(x) - 1))
              x <- dataWithLagNames(object)
              print(x)
              ## cat("\n")
          }
          )
\end{verbatim}

\begin{verbatim}
## TODO: "show", "Lagged2d"
\end{verbatim}


\begin{verbatim}
setMethod("show", "Lagged3d",
          function(object){
              .reportClassName(object, "Lagged3d")
              cat("Slot *data*:", "\n")

              ## x <- object@data
              ## if(is.null(dimnames(x)) || length(dimnames(x)) == 0){
              ##     d <- dim(x)
              ##     dimnames(x) <- list(rep("", d[1]), rep("", d[2]),
              ##                         paste0("Lag_", 0:(d[3] - 1)) )
              ## }
              x <- dataWithLagNames(object)
              print(x)
              ## cat("\n")
          }
          )
\end{verbatim}

\begin{verbatim}
## Commenting out since causes trouble by precluding default methods from printing.
##
## setMethod("show", "Lagged",
##           function(object){
##               ## .reportClassName(object, "Lagged") # this is silly: never writes!
##               ## callNextMethod()
##               wrk <- object@data
##               cat("Slot *data*:", "\n")
##               .printVecOrArray(wrk)
##               cat("\n")
##               ## callNextMethod() # in case the object inherits from other classes
##               ##                  # unfortunately, it prints slot data again.
##           }
##           )

setMethod("show", "FlexibleLagged",
          function(object){
              .reportClassName(object, "FlexibleLagged")
              cat("Slot *data*:", "\n")
              show(object@data)
          }
          )
\end{verbatim}


\section{Further constructors for lagged objects}
\label{sec:org94c0c15}

Function \texttt{new()} can be used to create objects from the lagged classes.
In this section we define some functions to make this more convenient.

First, a function to convert objects from S3 class "acf" (created by \texttt{acf()}) to "Lagged":
\begin{verbatim}
acf2Lagged <- function(x){
    acv <- x$acf
    d <- dim(acv)
    if(d[2] == 1 && d[3] == 1){
        data <- as.vector(acv)
        if(x$type == "partial") # lag-0 is missing, insert it
            data <- c(1, data)
        new("Lagged1d", data = data)
    }else{
        ## transpose to make the 3rd index corresponding to lag.
        ##   (taken from acfbase2sl() in package pcts, see the comments there)
        ##
        ## TODO: test!
        ## Note: in pcts:::acfbase2sl() the analogous command is aperm(acv, c(3,2,1))
        ##       i.e. R[k] is transposed => check if that is correct!
        data <- aperm(acv, c(2, 3, 1))

        if(x$type == "partial"){ # lag-0 is missing, insert it
            datanew <- array(NA_real_, dim(data) + c(0,0,1) )
            datanew[ , , -1] <- data
            data <- datanew
        }

        new("Lagged3d", data = data)
    }
}
\end{verbatim}


Function "Lagged" looks at the supplied data argument and chooses an appropriate class
inheriting from "Lagged". \textbf{TODO:} Make \texttt{Lagged()} generic?
\begin{verbatim}
Lagged <- function(data, ...){
    if(is.vector(data)){
        new("Lagged1d", data = data, ...)
    }else if(is.matrix(data)){
        new("Lagged2d", data = data, ...)
    }else if(is.array(data)){
        new("Lagged3d", data = data, ...)
    }else if(is(data, "Lagged")){
        new("FlexibleLagged", data = data, ...)
    }else if(inherits(data, "acf")){    # for S3 class "acf"
        acf2Lagged(data)
    }else
        stop("Cannot create a Lagged object from the given data")
}
\end{verbatim}

\textbf{TODO:} Tests!

\section{New}
\label{sec:org732c847}

The functions in this section are temporarily here during development and should be move to
more appropriate places eventually.

**

provideDimnames is new in R-3.0.0

\texttt{dataWithLagNames(object)} is a convenience function which works like
\texttt{object[]} but also ensures that the lag dimension has names. It is exported for use in other
packages. Occasionally users may wish to use it too.
\begin{verbatim}
dataWithLagNames <- function(object, prefix = "Lag_"){
    x <- object[]
    if(is.array(x)){
        d <- dim(x)
        nd <- length(d)

        xwithnams <- provideDimnames(x, base = list(""), unique = FALSE)
        dimnames(xwithnams)[[nd]] <- paste0(prefix, 0:(d[nd] - 1))
        xwithnams
    }else{
        if(is.null(names(x)) || length(names(x)) == 0)
            names(x) <- paste0(prefix, 0:(length(x) - 1))
        x
    }
}
\end{verbatim}
\end{document}
